\documentclass[10pt,a4paper]{article}
\usepackage[margin=2.5cm]{geometry}
\usepackage{enumitem}
\usepackage{hyperref}

\begin{document}

\centerline{\huge\textbf{Asa Gilmore}}
\vspace{0.5em}
\centerline{Univeristy of Washington, Neuroinformatics Research and
Development Group}
\centerline{asagil@uw.edu • +1 206-886-6950}

\section*{Education}
\textbf{In Progress: Bachelor's Of Science} in Mathematics \hfill 2023-Present \\
University of Washington

\section*{Publications}
\begin{enumerate}[leftmargin=*]
    \item Aahana Basappa, Pranay Goel, Anusri Karra, Anish Karra, \textbf{Asa Gilmore}, Kevin Zhu. (2025).
AMVICC: A Novel Benchmark for Cross-Modal Failure Mode Profiling for VLMs and IGMs. NeurIPS 2025, 1st Workshop on VLM4RWD.
https://openreview.net/forum?id=vDkdG55rDL.

    \item John Kruper, McKenzie P Hagen, François Rheault, Isaac Crane,
    \textbf{Asa Gilmore}, Manjari Narayan, Keshav Motwani,
Eardi Lila, Chris Rorden, Jason D Yeatman, Ariel Rokem. (2024). Tractometry
of the Human Connectome Project: resources and insights. Frontiers in Neuroscience,
Volume 18, https://doi.org/10.3389/fnins.2024.1389680.

\item John Kruper, Adam Richie-Halford, Joanna Qiao,
\textbf{Asa Gilmore}, Kelly Chang, Mareike Grotheer, Ethan Roy,
Sendy Caffara, Teresa Gomez, Sam Chou, Matthew Cieslak,
Serge Koudoro, Eleftherios Garyfallidis, Theodore D. Sattherthwaite,
Jason D. Yeatman, Ariel Rokem. (2025). A software ecosystem for brain tractometry
processing, analysis, and insight. PLOS Computational Biology, https://doi.org/10.1371/journal.pcbi.1013323
\end{enumerate}
\section*{Works in Progress}
\begin{enumerate}[leftmargin=*]
    \item \textbf{Asa Gilmore}, Anita Esi Eshun, Yue Wu, Aaron Lee, Ariel Rokem. (2025).
    Vessels hiding in plain sight: quantifying brain vascular morphology in anatomical MR images using deep learning.
    BioRxiv, https://doi.org/10.1101/2025.05.06.652518


\end{enumerate}

\section*{Research Experience}

\begin{description}
    \item[\textbf{Undergraduate Researcher}] at UW Neuroinformatics R\&D Group \hfill September 2023 - current
    \begin{itemize}[leftmargin=*]
        \item Implemented Parallel Tractography and Diffusion modeling in open source Diffusion analysis libraries DIPY, and pyAFQ.
        \item Developed a deep learning based Vessel Segmentation tool for anatomical MRI imaging.
        \item Currently developing a volumetric foundation model for neuroimaging data, utilizing a self supervised transformer model.
    \end{itemize}
    \item[\textbf{Machine Learning Researcher}] at Algoverse \hfill June 29th 2025 - September 31st 2025
    \\Led teams of four on machine learning research projects such as:
    \begin{itemize}[leftmargin=*]
        \item utilizing Adversarial autoencoders for scanner harmonization on Tractography data.
        \item Applying confidence calibration methods to Anat2Vess
    \end{itemize}
\end{description}

\section*{Awards}
\begin{itemize}[leftmargin=*]
    \item Awarded \$50,023 in compute by NAIRR to train a neuroimaging foundation model,
see (NAIRR240391), (2024)
    \item Awarded \$10,000 in compute by the University of Washington eScience institute for exploratory research on
    model architectures for volumetric foundation models in neuroimaging.
\end{itemize}

% \begin{small}
%     \section*{Skills}

%     \begin{itemize}[leftmargin=*]
%         \item pyTorch, Tensorflow, git, bash, linux, python, numpy, distributed computing,
%         AWS cloud, web dev, slurm
%         \item Experience with transformer models, computer vision, self supervised learning,
%         volumetric imaging, and medical image processing
%         \item LaTeX, Mathematics Typesetting, academic writing
%     \end{itemize}
% \end{small}
\end{document}